\def\duedate{08/31/2022}
\def\HWnum{1}
% Document setup
\documentclass[12pt]{article}
\usepackage[margin=1in]{geometry}
\usepackage{fancyhdr}
\usepackage{lastpage}

\pagestyle{fancy}
\lhead{Richard Whitehill}
\chead{MATH 757 -- HW \HWnum}
\rhead{\duedate}
\cfoot{\thepage \hspace{1pt} of \pageref{LastPage}}

% Encoding
\usepackage[utf8]{inputenc}
\usepackage[T1]{fontenc}

% Math/Physics Packages
\usepackage{amsmath}
\usepackage{amssymb}
\usepackage{mathtools}
\usepackage[arrowdel]{physics}
\usepackage{siunitx}

\AtBeginDocument{\RenewCommandCopy\qty\SI}

% Reference Style
\usepackage{hyperref}
\hypersetup{
    colorlinks=true,
    linkcolor=blue,
    filecolor=magenta,
    urlcolor=cyan,
    citecolor=green
}

\newcommand{\eref}[1]{Eq.~(\ref{eq:#1})}
\newcommand{\erefs}[2]{Eqs.~(\ref{eq:#1})--(\ref{eq:#2})}

\newcommand{\fref}[1]{Fig.~\ref{fig:#1}}
\newcommand{\frefs}[2]{Figs.~\ref{fig:#1}--\ref{fig:#2}}

\newcommand{\tref}[1]{Table~\ref{tab:#1}}
\newcommand{\trefs}[2]{Tables~\ref{tab:#1}-\ref{tab:#2}}

% Figures and Tables 
\usepackage{graphicx}
\usepackage{float}

\newcommand{\bef}{\begin{figure}[h!]\begin{center}}
\newcommand{\eef}{\end{center}\end{figure}}

\newcommand{\bet}{\begin{table}[h!]\begin{center}}
\newcommand{\eet}{\end{center}\end{table}}

% tikz
\usepackage{tikz}
\usetikzlibrary{calc}
\usetikzlibrary{decorations.pathmorphing}
\usetikzlibrary{decorations.markings}
\usetikzlibrary{arrows.meta}
\usetikzlibrary{positioning}

% tcolorbox
\usepackage[most]{tcolorbox}
\usepackage{xcolor}
\usepackage{xifthen}
\usepackage{parskip}

\newcommand*{\eqbox}{\tcboxmath[
    enhanced,
    colback=black!10!white,
    colframe=black,
    sharp corners,
    size=fbox,
    boxsep=8pt,
    boxrule=1pt
]}

% Miscellaneous Definitions/Settings
\newcommand{\prob}[2]{\textbf{#1)} #2}
\newcommand{\reals}{\mathbb{R}}
\newcommand{\integers}{\mathbb{Z}}
\newcommand{\naturals}{\mathbb{N}}
\newcommand{\rationals}{\mathbb{Q}}
\newcommand{\complexs}{\mathbb{C}}

\setlength{\parskip}{\baselineskip}
\setlength{\parindent}{0pt}
\setlength{\headheight}{14.49998pt}
\addtolength{\topmargin}{-2.49998pt}







\begin{document}
    
\textbf{\large Lecture 1}

\prob{1}{Estimate 
\begin{eqnarray}
    \label{eq:estimate-question}
    \Big| \frac{u(t+h) - u(t-h) }{2h} - u'(t) \Big| \leq~?
.\end{eqnarray}
for $h>0$ and $|\partial^3u(s)| \leq m$, $s \in [t-h,t+h]$.
}

We have the expression
\begin{eqnarray}
    \label{eq:u(t+h)}
    u(t+h) = u(t) + hu'(t) + \frac{h^2}{2}u''(t) + \int_{t}^{t+h} \frac{(t+h-s)^{2}}{2} u^{(3)}(s)\,\dd{s}
.\end{eqnarray}
Taking $h \rightarrow -h$, \eref{u(t+h)} becomes
\begin{eqnarray}
    \label{eq:u(t-h)}
    u(t-h) = u(t) - hu'(t) + \frac{h^2}{2}u''(t) - \int_{t-h}^{t} \frac{(t-h-s)^{2}}{2} u^{(3)}(s)\,\dd{s}
.\end{eqnarray}
Subtracting the two we get
\begin{align}
    \label{eq:subtract}
    u(t+h) - u(t-h) = 2hu'(t) + &\Bigg( \int_{t}^{t+h} \frac{(t+h-s)^{2}}{2} u^{(3)}(s)\,\dd{s} \notag \\
                                &- \int_{t-h}^{t} \frac{(t-h-s)^{2}}{2} u^{(3)}(s)\,\dd{s} \Bigg)
.\end{align}
Thus,
\begin{align}
    \label{eq:abs-value-expr}
    \Big| \frac{u(t+h) - u(t-h)}{2h} - u'(t) \Big| = \frac{1}{2h}&\Bigg| \int_{t}^{t+h} \frac{(t+h-s)^{2}}{2} u^{(3)}(s)\,\dd{s} \notag \\
    &- \int_{t-h}^{t} \frac{(t-h-s)^{2}}{2} u^{(3)}(s)\,\dd{s} \Bigg|
.\end{align}
Using $|a \pm b| \leq |a| + |b|$, $|\int f(x) \, \dd{x}| \leq \int |f(x)|\,\dd{x}$, and the condition $|u^{(3)}| \leq m$, we can write
\begin{align}
    \Big| \frac{u(t+h) - u(t-h)}{2h} - u'(t) \Big| &\leq \frac{m}{4h}\Big[\int_{t}^{t+h} (t+h-s)^{2}\, \dd{s} + \int_{t-h}^{t} (t-h-s)^2\, \dd{s} \Big] \notag \\
                                                   &= \frac{m}{4h}\Big( \frac{h^3}{3} + \frac{h^3}{3} \Big) = \eqbox{\frac{mh^2}{6}}
.\end{align}


\prob{2}{Let there be positive numbers $c,M$ such that for all $N \in \mathbb{Z}_{\geq 0}$
\begin{eqnarray}
    \label{eq:condition}
    |\partial^{N}u(s)| \leq c M^{N} N!, \qquad |s-a| \leq |t-a|
.\end{eqnarray}
Using
\begin{eqnarray}
    \label{eq:taylor-series}
    u(t) = \sum_{n=0}^{N-1} \frac{(t-a)^{n}}{n!} \partial^{n}u(a) + \int_{a}^{t} \frac{(t-s)^{N-1}}{(N-1)!} \partial^{N} u(s)\, {\rm d}s
,\end{eqnarray}
show that 
\begin{eqnarray}
    \label{eq:exercise2-res}
    \Big|u(t) - \sum_{n=0}^{N-1} \frac{(t-a)^{n}}{n!}\partial^{n}u(a) \Big| \leq c(M|t-a|)^{N} \rightarrow 0
\end{eqnarray}
if $M|t-a| < 1$. In this case we obtain Taylor's series for function $u$.
}

From \eref{taylor-series} we have (assuming $t > a$)
\begin{align}
    \label{eq:taylor-series-abs-val}
    \Bigg| u(t) - \sum_{n=0}^{N-1} \frac{(t-a)^{n}}{n!} \partial^{n}u(a) \Bigg| &= \Bigg| \int_{a}^{t} \frac{(t-s)^{N-1}}{(N-1)!} \partial^{N} u(s)\, {\rm d}s \Bigg| \notag \\
    &\leq \frac{c M^{N}N!}{(N-1)!} \int_{a}^{t} |t-s|^{N-1}\, \dd{s} \notag \\
    &= c N M^{N} \frac{(t-s)^{N}}{N}\Bigg|_{a}^{t} \notag \\
    &= c M^{N}[-(a-s)^{N}] \notag\\
    &= \eqbox{c\Big[M|t-a|\Big]^{N}}
.\end{align}
Note that the argument is similar for $t < a$. 
We get a relative $-$ sign from switching the bounds of integration and another from having $t < s$ such that $|t-s| = -(t-s)$.
These minus signs cancel giving us the result in \eref{taylor-series-abs-val}.

If $M|t-a|<1$, then in the limit $N\rightarrow\infty$, we have $c[M|t-a|]^{N} \rightarrow 0$.


\prob{3}{Let there be positive numbers $c,M$ such that for all $N \in \mathbb{Z}_{\geq 0}$
\begin{eqnarray}
    \label{eq:condition-3}
    |\partial^{N}u(s)| \leq cM^{N}(N!)^{\alpha}, \qquad \alpha<1,\quad s \in \mathbb{R}
.\end{eqnarray}
Show that in this case we have convergent Taylor series for all $t,a \in \mathbb{R}$.
}

Note that
\begin{eqnarray}
    \label{eq:condition-rearrange}
|\partial^{N}u(s)| \leq c M^{N} (N!)^{\alpha} = \frac{c M^{N} N!}{(N!)^{1-\alpha}}
.\end{eqnarray}
Thus, we can alter \eref{taylor-series-abs-val} to read
\begin{eqnarray}
    \label{eq:taylor-abs-val-3}
    \Bigg| u(t) - \sum_{n=0}^{N-1} \frac{(t-a)^{n}}{n!}\partial^{n}u(a)  \Bigg| \leq \frac{c[M|t-a|]^{N}}{(N!)^{1-\alpha}} \rightarrow 0, \quad \mbox{as } N \rightarrow \infty 
,\end{eqnarray}
noting that factorials grow faster than powers of $N$.

\textbf{\large Lecture 2}

\prob{1}{Check that
\begin{eqnarray}
    \label{eq:v(t)}
    v(t) = \int_{0}^{t} \frac{\sin[(t-\tau)\sqrt{A}]}{\sqrt{A}}f(\tau)\,\dd{\tau}
.\end{eqnarray}
satisfies $v(0)=\partial_{t}v(0)=0$ and $(\partial_{t}^{2}+A)v=f(t)$ for $t>0$.
}

It is clear that $v(0) = 0$ since the upper bound is the same as the lower bound at $t = 0$.
Next, 
\begin{align}
    \label{eq:derivative-at-0}
    \partial_{t}v(t) &= \frac{\sin{[(t-\tau)\sqrt{A}]}}{\sqrt{A}}f(\tau)\Bigg|_{\tau=t} + \int_{0}^{t}\partial_{t}\Big(\frac{\sin{[(t-\tau)\sqrt{A}]}}{\sqrt{A}}\Big)f(\tau)\, \dd{\tau} \notag \\
                     &= \int_{0}^{t} \cos{[(t-\tau)\sqrt{A}]}f(\tau)\,\dd{\tau} \Rightarrow \eqbox{\partial_{t}v(0) = 0}
.\end{align}
Finally,
\begin{align}
    \label{eq:v''}
    \partial_{t}^2v(t) &= \partial_{t} \int_{0}^{t} \cos{[(t-\tau)\sqrt{A}]}f(\tau)\, \dd{\tau} \notag \\
                       &= f(\tau) - \int_{0}^{t} \sqrt{A}\sin{[(t-\tau)\sqrt{A}]}f(\tau)\,\dd{\tau} \notag \\
                       &= f(\tau) - A v(t) \notag \\
                       &\Rightarrow \eqbox{(\partial_{t}^2-A)v(t) = f(\tau)}
\end{align}


\prob{2}{Check
\begin{align}
    \label{eq:u(t)-soln}
    u(t) = \frac{\sin{[(T-t)\sqrt{A}}]}{\sin{(T\sqrt{A})}}g &+ \frac{\sin{(t\sqrt{A})}}{\sin{(T\sqrt{A})}}\Big[ h - \int_{0}^{T} \frac{\sin{[(T-\tau)\sqrt{A}]}}{\sqrt{A}}f(\tau)\,\dd{\tau} \Big] \notag \\
                                                            &+ \int_{0}^{t} \frac{\sin{[(t-\tau)\sqrt{A}]}}{\sqrt{A}}f(\tau)\,\dd{\tau}
.\end{align}
satisfies the boundary conditions $u(0)=g$ and $u(T)=h$. 
}

First, we check at $t=0$:
\begin{eqnarray}
    \label{eq:check-0}
    u(0) = \frac{\sin{(T\sqrt{A})}}{\sin{(T\sqrt{A})}}g = g
.\end{eqnarray}
Next, we check at $t=T$:
\begin{eqnarray}
    \label{eq:check-T}
    u(T) = h - \int_{0}^{T} \frac{\sin[(T-\tau)\sqrt{A}]}{\sqrt{A}}f(\tau)\,\dd{\tau} + \int_{0}^{T} \frac{\sin[(T-\tau)\sqrt{A}]}{\sqrt{A}}f(\tau)\,\dd{\tau} = h
.\end{eqnarray}



\end{document}
