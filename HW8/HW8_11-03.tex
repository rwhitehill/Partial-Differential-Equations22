\def\duedate{11/03/22}
\def\HWnum{8}
% Document setup
\documentclass[12pt]{article}
\usepackage[margin=1in]{geometry}
\usepackage{fancyhdr}
\usepackage{lastpage}

\pagestyle{fancy}
\lhead{Richard Whitehill}
\chead{MATH 757 -- HW \HWnum}
\rhead{\duedate}
\cfoot{\thepage \hspace{1pt} of \pageref{LastPage}}

% Encoding
\usepackage[utf8]{inputenc}
\usepackage[T1]{fontenc}

% Math/Physics Packages
\usepackage{amsmath}
\usepackage{amssymb}
\usepackage{mathtools}
\usepackage[arrowdel]{physics}
\usepackage{siunitx}

\AtBeginDocument{\RenewCommandCopy\qty\SI}

% Reference Style
\usepackage{hyperref}
\hypersetup{
    colorlinks=true,
    linkcolor=blue,
    filecolor=magenta,
    urlcolor=cyan,
    citecolor=green
}

\newcommand{\eref}[1]{Eq.~(\ref{eq:#1})}
\newcommand{\erefs}[2]{Eqs.~(\ref{eq:#1})--(\ref{eq:#2})}

\newcommand{\fref}[1]{Fig.~\ref{fig:#1}}
\newcommand{\frefs}[2]{Figs.~\ref{fig:#1}--\ref{fig:#2}}

\newcommand{\tref}[1]{Table~\ref{tab:#1}}
\newcommand{\trefs}[2]{Tables~\ref{tab:#1}-\ref{tab:#2}}

% Figures and Tables 
\usepackage{graphicx}
\usepackage{float}

\newcommand{\bef}{\begin{figure}[h!]\begin{center}}
\newcommand{\eef}{\end{center}\end{figure}}

\newcommand{\bet}{\begin{table}[h!]\begin{center}}
\newcommand{\eet}{\end{center}\end{table}}

% tikz
\usepackage{tikz}
\usetikzlibrary{calc}
\usetikzlibrary{decorations.pathmorphing}
\usetikzlibrary{decorations.markings}
\usetikzlibrary{arrows.meta}
\usetikzlibrary{positioning}

% tcolorbox
\usepackage[most]{tcolorbox}
\usepackage{xcolor}
\usepackage{xifthen}
\usepackage{parskip}

\newcommand*{\eqbox}{\tcboxmath[
    enhanced,
    colback=black!10!white,
    colframe=black,
    sharp corners,
    size=fbox,
    boxsep=8pt,
    boxrule=1pt
]}

% Miscellaneous Definitions/Settings
\newcommand{\prob}[2]{\textbf{#1)} #2}
\newcommand{\reals}{\mathbb{R}}
\newcommand{\integers}{\mathbb{Z}}
\newcommand{\naturals}{\mathbb{N}}
\newcommand{\rationals}{\mathbb{Q}}
\newcommand{\complexs}{\mathbb{C}}

\setlength{\parskip}{\baselineskip}
\setlength{\parindent}{0pt}
\setlength{\headheight}{14.49998pt}
\addtolength{\topmargin}{-2.49998pt}







\begin{document}
    
\prob{6.1.2}{
Show that a function which is a power series in the complex variable $x + iy$ must satisfy the Cauchy-Riemann equations and therefore Laplace's equations.
}

We define a power series
\begin{eqnarray}
    \label{eq:power-series}
    f(z) = \sum_{n=0}^{\infty} a_{n} z^{n} 
.\end{eqnarray}
We can separate the sum into real and imaginary parts using the binomial series since $z = x+iy$
\begin{align}
    \label{eq:rewrite-power-series}
    f(z) &= \sum_{n=0}^{\infty} a_{n} (x+iy)^{n} = \sum_{n=0}^{\infty} a_{n} \sum_{k=0}^{n} \binom{n}{k} x^{n-k}(iy)^{k} \\
    &= \sum_{n=0}^{\infty} \Big[ a_{n} \sum_{k=0}^{m} \binom{n}{2k} (-1)^{k} x^{n-2k}y^{2k} + i\sum_{k=0}^{m-o(n)} \binom{n}{2k+1} (-1)^{k} x^{n-2k-1}y^{2k+1} \Big]
,\end{align}
where $\displaystyle m = \begin{cases} n/2 & \mbox{if}~n~\mbox{is even} \\ (n-1)/2 & \mbox{if}~n~\mbox{is odd} \end{cases}$ and $\displaystyle o(n) = \begin{cases} 1 & \mbox{if}~n~\mbox{is even} \\ 0 & \mbox{if}~n~\mbox{is odd} \end{cases}$.
If we let 
\begin{eqnarray}
\label{eq:u-v-def}
\begin{aligned}
    u(x,y) &= \sum_{n=0}^{\infty} a_{n} \sum_{k=0}^{m} \binom{n}{2k} (-1)^{k} x^{n-2k}y^{2k} \\
    v(x,y) &= \sum_{n=0}^{\infty} a_{n} \sum_{k=0}^{m-o(n)} \binom{n}{2k+1} (-1)^{k} x^{n-2k-1}y^{2k+1}
.\end{aligned}
\end{eqnarray}
All that remains now is to check that the Cauchy-Riemann conditions are satisfied.
\begin{subequations}    
\begin{eqnarray}
    \label{eq:CR-conds-6.1.2-1}
    \begin{aligned}
        u_{x} &=  \sum_{n=0}^{\infty} a_{n} \sum_{k=0}^{m-o(n)} \binom{n}{2k} (-1)^{k} (n-2k)x^{n-2k-1}y^{2k} \\
        &= \sum_{n=0}^{\infty} a_{n} \sum_{k=0}^{m-o(n)} (-1)^{k} \frac{n!}{(2k)!(n-2k-1)!}x^{n-2k-1}y^{2k} \\
        v_{y} &= \sum_{n=0}^{\infty} a_{n} \sum_{k=0}^{m-o(n)} \binom{n}{2k+1} (-1)^{k} (2k+1) x^{n-2k-1}y^{2k} \\
        &= \sum_{n=0}^{\infty} a_{n} \sum_{k=0}^{m-o(n)} (-1)^{k} \frac{n!}{(2k)!(n-2k-1)!} x^{n-2k-1}y^{2k}
    .\end{aligned}
\end{eqnarray}
Note that the $o(n)$ term in the upper index of the $u_{x}$ series comes from the fact that when $n$ is even, the $k = m$ terms gives 0 when differentiated (with respect to $x$) since $x^{n - 2m} = 1$.
\begin{eqnarray}
    \label{eq:CR-conds-6.1.2-2}
    \begin{aligned}
        u_{y} &= \sum_{n=0}^{\infty} a_{n} \sum_{k=1}^{m} (-1)^{k} \frac{n!}{(2k-1)!(n-2k)!} x^{n-2k}y^{2k-1} \\
        v_{x} &= \sum_{n=0}^{\infty} a_{n} \sum_{k=1}^{m-o(n)+(o(n)-1)} (-1)^{k} \frac{n!}{(2k+1)!(n-2(k-1))!} x^{n-2(k-1)}y^{2k+1} \\
        &= -\sum_{n=0}^{\infty} a_{n} \sum_{k=1}^{m} (-1)^{k} \frac{n!}{(2k-1)!(n-2k)!} x^{n-2k}y^{2k-1}
    ,\end{aligned}
\end{eqnarray}
\end{subequations}
where the $o(n) - 1$ term arises in the $v_{x}$ term since $n-2k-1 = 0$ if $k = m$ whenever $n$ is odd.
Thus, we can see that any arbitrary power series in $z$ satisfies the Cauchy-Riemann conditions.

\prob{6.1.5}{
    Solve $u_{xx} + u_{yy} = 1$ in $r < a$ with $u(x,y)$ vanishing on $r = a$.
}

The equation $u_{xx} + u_{yy} = \Delta u = 1$ is stated in cartesian coordinates.
We could equivalently pose the problem in polar coordinates in terms of $r$ and $\theta$, which appears as follows:
\begin{eqnarray}
    \label{eq:equiv-polar}
    \Delta u = \frac{1}{r}\partial_{r}(r\partial_{r} u) + \frac{1}{r}\partial_{r} u + \frac{1}{r^2}\partial_{\theta}^2 u = 1
.\end{eqnarray}
Assuming that $u(r,\theta)=u(r)$ (i.e. $u$ is rotationally invariant), then
\begin{eqnarray}
\label{eq:solve-eq}
\begin{aligned}
    \partial_{r}(r\partial_{r} u) &= r \\
    r\partial_{r} u &= \frac{r^2}{2} + c_{1} \\
    \partial_{r}u &= \frac{r}{2} + \frac{c_1}{r} \\
    u &= \frac{r^2}{4} + c_1 \ln{r} + c_2
.\end{aligned}
\end{eqnarray}
We must have $c_1 = 0$ since $\ln{r} \rightarrow \infty$ when $r \rightarrow 0$,
and imposing the boundary condition at $r = a$ we have
\begin{eqnarray}
    \label{eq:bound-cond}
    0 = \frac{a^2}{4} + c_2 \Rightarrow c_2 = - \frac{a^2}{4} 
.\end{eqnarray}
Hence, our solution is as follows:
\begin{eqnarray}
    \label{eq:sol-6.1.5}
    \eqbox{
    u = \frac{1}{4}[r^2 - a^2] 
}
.\end{eqnarray}


\prob{6.1.6}{
Solve $u_{xx} + u_{yy} = 1$ in the annulus $a < r < b$ with $u(x,y)$ vanishing both parts of the boundary $r = a$ and $r = b$.
}

From the last problem, we have the general solution 
\begin{eqnarray}
    \label{eq:gen-sol-u}
    u = \frac{r^2}{4} + c_1\ln{r} + c_2 
.\end{eqnarray}
This time, though, we keep the term proportional to $\ln{r}$ since it is well behaved on the domain of interest.
Plugging in boundary conditions we find
\begin{eqnarray}
    \label{eq:bound-cond-a-b}
    \begin{aligned}
        0 = \frac{a^2}{4} + c_1\ln{a} + c_2 \\
        0 = \frac{b^2}{4} + c_1\ln{b} + c_2
    \end{aligned}
.\end{eqnarray}
Solving the system of equations we find
\begin{align}
    \label{eq:sol-c1-c2}
    c_1 &= -\frac{b^2 - a^2}{4\ln{(b/a)}} \\
    c_2 &= \frac{b^2\ln{a} - a^2\ln{b}}{4\ln{(b/a)}}
.\end{align}
Thus, our solution is of the form
\begin{eqnarray}
    \label{eq:sol-u}
    \eqbox{
    u = \frac{1}{4} \Big[ r^2 - \frac{b^2 - a^2}{\ln{(b/a)}}\ln{r} + \frac{b^2\ln{a} - a^2\ln{b}}{\ln{(b/a)}} \Big]
}
.\end{eqnarray}


\prob{6.4.6}{
    Find the harmonic function $u$ in the semidisk $\{ r < 1, 0 < \theta < \pi \} $ with $u$ vanishing on the diameter ($\theta = 0,\pi$) and $u = \pi \sin{\theta} - \sin{2\theta}$ on $r = 1$.
}

We need to have $\Delta u = 0$.
In polar coordinates
\begin{eqnarray}
    \label{eq:polar-lap}
    \Delta u = \frac{1}{r}\partial_{r}(r\partial_{r}u) + \frac{1}{r^2}\partial_{\theta}^2 u = 0
.\end{eqnarray}
If we let $A = -\partial_{\theta}^2$ on $(0,\pi)$, where $u|_{\theta=0} = u|_{\theta=\pi} = 0$, then $A$ has eigenfunctions $e_{n}(\theta) = \sin{n\theta}$ ($||e_{n}||^2 = \pi/2$) with corresponding eigenvalues $n^2$.
In class we have solve the remaining equation $r^2\partial_{r}^2 u + r\partial_{r} u - Au = 0$:
\begin{eqnarray}
    \label{eq:solve-harmonic-u-A}
    u = r^{\sqrt{A}}c_1 + r^{-\sqrt{A}}c_2
.\end{eqnarray}
Since we need $u$ to be bounded on the domain, it follows that $c_2 = 0$.
Plugging in our boundary conditions we find
\begin{eqnarray}
    \label{eq:bound-cond-r1}
    u(1) = c_1 = \pi\sin{\theta} - \sin{2\theta} = g
.\end{eqnarray}
Hence, our solution is given as
\begin{eqnarray}
    \label{eq:sol-u-g}
    u(r,\theta) =  r^{\sqrt{A}} g = \sum_{n=1}^{\infty} r^{n} \hat{g}_{n} \sin{n\theta}
,\end{eqnarray}
where
\begin{eqnarray}
    \label{eq:gnhat}
    \hat{g}_{n} = \frac{2}{\pi} \int_{0}^{\pi} (\pi\sin{\theta} - \sin{2\theta})\sin{n\theta} \dd{\theta} = \pi \delta_{1n} - \delta_{2n}
.\end{eqnarray}
Note that $\delta_{nm}$ is the kronecker-delta symbol.
Hence, our solution simply becomes
\begin{eqnarray}
    \label{eq:sol-u-simplify}
    \eqbox{
    u(r,\theta) = \pi r \sin{\theta} - r^2\sin{2\theta}
}
.\end{eqnarray}




\prob{Lecture 15 - 1}{
    Show that $\partial z^{n} = nz^{n-1}$ and $\bar{\partial} \bar{z}^{n} = n \bar{z}^{n-1}$ for $n = 0,1,2,\ldots$.
}

We write
\begin{align}
    \label{eq:deriv-zn}
    \partial z^{n} &= \frac{1}{2} (\partial_{x} - i \partial_{y})(x + iy)^{n} = \frac{1}{2} [ \partial_{x}(x + iy)^{n} - i\partial_{y} (x + iy)^{n} ] \\
    &= \frac{1}{2}[n(x+iy)^{n-1} + n(x+iy)^{n-1}] = n (x+iy)^{n-1} = nz^{n-1}
.\end{align}
Similarly
\begin{align}
    \label{eq:deriv-zbarn}
    \bar{\partial} \bar{z}^{n} &= \frac{1}{2} (\partial_{x} + i \partial_{y})(x - iy)^{n} = \frac{1}{2} [ \partial_{x}(x - iy)^{n} + i\partial_{y} (x - iy)^{n} ] \\
    &= \frac{1}{2}[n(x-iy)^{n-1} + n(x-iy)^{n-1}] = n (x-iy)^{n-1} = n \bar{z}^{n-1}
.\end{align}


\prob{Lecture 15 - 2}{
    Find $\bar{\partial} |z|^2$, $\partial |z|^2$, and $\Delta |z|^2$.
}

We have
\begin{eqnarray}
    \label{eq:partialbar-z2}
    \eqbox{
    \bar{\partial} |z|^2 = \frac{1}{2}(\partial_{x} + i \partial_{y}) (x^2 + y^2) = \frac{1}{2}(2x + 2iy) = x + iy = z
}
.\end{eqnarray}
Similarly
\begin{eqnarray}
    \label{eq:partial-z2}
    \eqbox{
    \partial |z|^2 = \frac{1}{2}(\partial_{x} - i \partial_{y}) (x^2 + y^2) = x - iy = \bar{z}
}
.\end{eqnarray}
Thus,
\begin{eqnarray}
    \label{eq:lap-z2}
    \Delta |z|^2 = 4 \partial\bar{\partial} |z|^2 = 4 \partial z = 4
.\end{eqnarray}




\prob{Lecture 15 - 3}{
    For an analytic function $f$ in the open set $\Omega$, let $u(x,y) = \Re f(z)$ and $v(x,y) = \Im f(z)$.
    Show that $\Delta u = \Delta v = 0$ and find $\nabla u \cdot \nabla v$.
    Functions $u$ and $v$ are called harmonic conjugate functions.
}

Since $f$ is analytic $\bar{\partial}f = 0$, and furthermore $\Delta f = 4\partial\bar{\partial}f = 0$.
We know that the Laplace operator is linear, which implies that
\begin{eqnarray}
    \label{eq:lap-f-uv}
    \Delta f = \Delta u + i \Delta v = 0 
.\end{eqnarray}
Thus, in order to have this equation hold generally, we must have its real and imaginary parts be zero separately, giving us
\begin{eqnarray}
    \label{eq:uv-harmonic}
    \eqbox{
    \Delta u = \Delta v = 0 
}
.\end{eqnarray}

For the second expression we will use the Cauchy-Riemanan conditions (which is a necessary and sufficient for a function to be analytic).
We write
\begin{eqnarray}
    \label{eq:gradu-gradv}
    \eqbox{
    \nabla u \cdot \nabla v = u_{x}v_{x} + u_{y}v_{y} = -u_{x}u_{y} + u_{y}u_{x} = 0
}
.\end{eqnarray}



\prob{Lecture 15 - 4}{
Using the divergence theorem rewrite the integral
\begin{eqnarray}
    \label{eq:15-4-int}
    \int_{\Omega} \overline{\partial} f(z) \dd{x}\dd{y}, \qquad f \in C^{1}(\Omega_{1}),~\Omega_{1} \supset \Omega
\end{eqnarray}
as an integral over the boundary $\partial \Omega$ of the domain $\Omega$.
}

We can write $f(z) = u(x,y) + iv(x,y)$, meaning
\begin{align}
    \label{eq:int-rewrite}
    \int_{\Omega} \bar{\partial}f(z) \dd{x}\dd{y} &= \frac{1}{2} \int_{\Omega} (\partial_{x} + i \partial_{y})f \dd{x}\dd{y} \notag \\
    &= \frac{1}{2} \int_{\Omega} [\partial_{x} f - \partial_{y} [-if]] \dd{x}\dd{y} \notag \\
    &= \frac{1}{2} \int_{\partial \Omega} [-if \dd{x} + f\dd{y}] \notag \\
    &= \frac{1}{2i} \int_{\partial \Omega} f[\dd{x} + i\dd{y}] \notag \\
    &= \eqbox{ \frac{1}{2i} \int_{\partial \Omega} f(z) \dd{z} }
.\end{align}
If $f$ is analytic, then $\bar{\partial}f = 0$, and the integral is zero.
That is,
\begin{eqnarray}
    \label{eq:closed-loop-integral}
    \int_{\Omega} \bar{\partial}f(z) \dd{x}\dd{y} = \frac{1}{2i} \int_{\partial \Omega} f(z) \dd{z} = 0
.\end{eqnarray}



\prob{Lecture 15 - 5}{
For harmonic conjugate functions $u$ and $v$ from Exercise 3 above show that $u_{x} = v_{y}$ and $u_{y} = -v_{y}$ (Caucy-Riemann equations).
}

An analytic function $f = u + iv$ satisfies $\bar{\partial}f = 0$, which implies that
\begin{eqnarray}
    \label{eq:implication-1}
    \frac{1}{2}(\partial_{x} + i \partial_{y})(u+iv) = \frac{1}{2}([\partial_{x} u - \partial_{y}v] + i [\partial_{y}u + \partial_{x}v]) = 0
.\end{eqnarray}
For this equality to hold true we must have that the real and imaginary parts of the middle expression are identically zero individually, meaning that
\begin{eqnarray}
    \label{eq:Cauchy-Riemann}
    \eqbox{
    \begin{aligned}
        u_{x} &= v_{y} \\
        u_{y} &= -v_{x}
    \end{aligned} 
}
.\end{eqnarray}



\end{document}
