\def\duedate{09/08/2022}
\def\HWnum{2}
\input{../preamble.tex}

\begin{document}
    
\prob{1.1.2}{Which of the following operators are linear?}

(a) $\displaystyle \mathcal{L}u = u_{x} + x u_{y}$

Linear operators of order $n$ are of the form 
\begin{eqnarray}
    \label{eq:nth-lin-op}
    \mathcal{L} = \sum_{m=0}^{n} \sum_{k=0}^{m} a_{m k}(x,y) \partial_{x}^{m-k} \partial_{y}^{k}
,\end{eqnarray}
where we assume that mixed derivatives can be interchanged according to Schwarz's theorem and the functions $a_{m k}(x,y)$ serve as coefficients for the derivative terms.

For $n = 1$, we see that \eref{nth-lin-op}, reduces to
\begin{eqnarray}
    \label{eq:lin-op-1}
    \mathcal{L} = a_{00}(x,y) + a_{10}(x,y)\partial_{x} + a_{11}\partial_{y}
.\end{eqnarray}
It is clear that $a_{00} \equiv 0$, $a_{10} = 1$ and $a_{11} = 1$, so this is a linear operator.

(b) $\displaystyle \mathcal{L}u = u_{x} + u u_{y}$

This is not a linear operator since it involves a product of $u$ and a derivative of $u$, which does not match the form of \eref{nth-lin-op}.
We can also show that it does not satisfy the property of linear scaling ($\mathcal{L}(\alpha u) = \alpha \mathcal{L} u$):
\begin{eqnarray}
    \label{eq:disprove-b}
    (\alpha u)_{x} + (\alpha u)(\alpha u)_{y} = \alpha \left( u_{x} + \alpha (u u_{y}) \right) \ne \alpha (u_{x} + u u_{y})
.\end{eqnarray}


(c) $\displaystyle \mathcal{L}u = u_{x} + u_{yy}$

This is clearly a linear operator since it matches the form given in \eref{nth-lin-op} for $n=2$ and also since any order of partial derivatives is linear.

(d) $\displaystyle \mathcal{L}u = u_{x} + u_{y} + 1$

This is linear since it matches the form \eref{nth-lin-op} for $n=1$, explicitly listed in \eref{lin-op-1}

(e) $\displaystyle \mathcal{L}u = \sqrt{1 + x^2}(\cos{y})u_{x} +  - u_{yxy} - [\arctan(x/y)]u$

This is a linear operator since this is a linear combination of derivatives of $u$ and $u$ itself, matching the form of \eref{nth-lin-op}


\prob{1.2.9}{Solve the equation $\displaystyle u_{x} + u_{y} = 1$}

We can make the change of variables $(x,y) \rightarrow (x',y')$ as follows:
\begin{align}
    \label{eq:change-var}
    x' = x + y \\
    y' = x - y
.\end{align}
Thus, we have
\begin{align}
    \label{eq:diff-change} 
    u_{x} = \pdv{x'}{x}u_{x'} + \pdv{y'}{x}u_{y'} = u_{x'} + u_{y'} \\
    u_{y} = \pdv{x'}{y}u_{x'} + \pdv{y'}{y}u_{y'} = u_{x'} - u_{y'}
.\end{align}

Substituting into the original PDE we have
\begin{eqnarray}
    \label{eq:changed-eq}
    u_{x} + u_{y} = 2 u_{x'} = 1     
.\end{eqnarray}
Solving, we find
\begin{eqnarray}
    \label{eq:solution-2}
    \eqbox{
    u(x,y) = \frac{1}{2}x' + f(y') = \frac{1}{2}(x + y) + f(x - y)
    }
,\end{eqnarray}
where the constant function is the solution to the homogeneous equation $u_{x} + u_{y} = 0$, and the first term is a solution to the nonhomogeneous equation.



\prob{1.3.9}{
This is an exercise on the divergence theorem
\begin{eqnarray}
    \label{eq:div-thm}
    \iiint_{D} \div{\va*{F}} \dd{\va*{x}} = \iint_{\partial D} \va*{F} \cdot \vu*{n} \dd{S}
\end{eqnarray}
valide for any bounded domain $D$ in space with boundary surface $\partial D$ and unit outward normal vector $\vu*{n}$.
As an exercise, verify it in the following case by calculating both sides separately: $\va*{F} = r^2 \va*{x},~ \va*{x} = x \vu*{x} + y \vu*{y} + z \vu*{z},~ r^2 = x^2 + y^2 + z^2$ and $D = $ the ball of radius $a$ and center at the origin.
}

We have $D = \{ \va*{x}~:~|\va*{x} < a| \}$ and $\partial D = \{ \va*{x}~:~|\va*{x}|=a \}$.
Looking at the first integral, we have
\begin{align}
    \label{eq:first-int}
    \iiint_{D} \div{\va*{F}} \dd{\va*{x}} &= \iiint_{D} [(2x^2 + r^2) + (2y^2 + r^2) + (2z^2 + r^2)] \dd{\va*{x}} \notag \\
                                          &= \iiint_{D} 5r^2 \dd{\va*{x}} = \int_{0}^{\pi}\int_{0}^{2\pi}\int_{0}^{a} 5r^2 (r^2 \sin{\theta})\dd{r}\dd{\phi}\phi{\theta} \notag \\
                                          &= (\pi)(2\pi)a^{5} = \eqbox{4 \pi a^{5}}
.\end{align}
Turning to the second integral now
\begin{align}
    \label{eq:sec-int}
    \iint_{\partial D} \va*{F} \cdot \vu*{n}\Bigr|_{\partial D} \dd{S} &= \int_{0}^{\pi}\int_{0}^{2\pi} a^3 (a^{2} \sin{\theta})\dd{\phi}\dd{\theta} \notag \\
                                                                       &= \eqbox{4\pi a^{5}}
.\end{align}
Since the results of both integrals match, we have shown the divergence theorem to be valid in this case.

\prob{1.3.10}{
If $\va*{f}(\va*{x})$ is continuous and $|\va*{f}(\va*{x})| \leq 1/(|\va*{x}|^3 + 1)$ for all $\va*{x}$, show that
\begin{eqnarray}
    \label{eq:3_10-result}
    \iiint_{\reals^3} \div{\va*{f}} \dd{\va*{x}} = 0 
.\end{eqnarray}

}

We can write
\begin{eqnarray}
    \label{eq:int-limit}
    \int_{\reals^{3}} \div{\va*{f}(\va*{x})} \dd{\va*{x}} = \lim_{a \rightarrow \infty} \int_{B(a)} \grad{\va*{f}(\va*{x})} \dd{\va*{x}} = \lim_{a \rightarrow \infty} \int_{\partial B(a)} \va*{f}(\va*{x}) \cdot \vu*{r}\Bigr|_{\partial B(a)} a^2 \dd{\Omega}
,\end{eqnarray}
where $B(a)$ is an open sphere of radius $a$ and $\partial B(a)$ is the surface of the sphere of radius $a$, and $\dd{\Omega} = \sin{\theta} \dd{\phi}\dd{\theta}$.

Using some basic integral and dot product inequalities we find that
\begin{eqnarray}
    \label{eq:inequality-int}
    \left| \int_{\partial B(a)} \va*{f} \cdot \vu*{r} \dd{\Omega} \right| \leq \int_{\partial B(a)} |\va*{f} \cdot \vu*{r}| \dd{\Omega} \leq \int_{\partial B(a)} |\va*{f}| |\vu*{r}| \dd{\Omega}
.\end{eqnarray}

Using the assumption that $\va*{f}$ is bounded at each $\va*{x}$ in the manner specified above, we find
\begin{eqnarray}
    \label{eq:inequality-f}
    \int_{\partial B(a)} |\va*{f}|\Bigr|_{\partial B(a)} \dd{\Omega} \leq \int_{0}^{\pi}\int_{0}^{2\pi} \frac{4\pi}{a^3 + 1}
.\end{eqnarray}
Hence,
\begin{eqnarray}
    \label{eq:result-3_10}
    \eqbox{
        \left| \int_{\reals^3} \div{\va*{f}(\va*{x})} \dd{\va*{x}} \right| \leq \lim_{a \rightarrow \infty} \frac{4 \pi a^2}{a^3 + 1} = 0 \Rightarrow \int_{\reals^3} \div{\va*{f}(\va*{x})} \dd{\va*{x}} = 0
    } 
.\end{eqnarray}

\prob{2.1.1}{
    Solve $u_{tt} = c^2 u_{xx},~u(x,0)=e^{x},~u_{t}(x,0) = \sin{x}$.
}

We know the general solution of the wave equation with initial conditions $u(x,0) = \phi(x)$ and $u_{t}(x,0) = \psi(x)$ is 
\begin{eqnarray}
    \label{eq:wave-initial-cond-sol}
    u(x,t) = \frac{1}{2}\left[ \phi(x+ct) + \phi(x-ct) \right] + \frac{1}{2c}\int_{x - ct}^{x + ct} \psi(s) \dd{s}
.\end{eqnarray}
In this case, we have $\phi(x) = e^{x}$ and $\psi(x) = \sin{x}$, so
\begin{align}
    \label{eq:wave-soln}
    u(x,t) &= \frac{1}{2}\left[ e^{x + ct} + e^{x - ct} \right] + \frac{1}{2c}\int_{x-ct}^{x+ct} \sin{s} \dd{s} \notag \\
           &= \frac{1}{2}e^{x}\left[ e^{ct} + e^{-ct} \right] - \frac{1}{2c}\left[ \cos{(x+ct)} - \cos{(x-ct)} \right] \notag \\
           &= e^{x} \cosh{ct} + \frac{1}{c}\sin{x}\sin{ct}
,\end{align}
where we have used the definition of $\cosh$ and the relation $\cos(a+b) - \cos(a-b) = \\ -2\sin{a}\sin{b}$ to simplify our result. 

\prob{2.1.8}{
    A \textit{spherical wave} is a solution of the three-dimensional wave equation of the form $u(r,t)$, where $r$ is the distance to the origin (the spherical coordinate).
    The wave equation takes the form
    \begin{eqnarray}
        \label{eq:sphere-wave-eq}
        u_{t t} = c^2\left( u_{rr} + \frac{2}{r} u_{r} \right)
    .\end{eqnarray}
}

(a) Change variables $v = ru$ to get the equation for v: $v_{t t} = c^2 v_{rr}$.

If we make the change of variables $v = ru$, we obtain
\begin{align}
    \label{eq:change-var-sphere}
    u_{r} &= \left( \frac{v}{r} \right)_{r} = -\frac{1}{r^2}v + \frac{1}{r}v_{r} \\
    u_{rr} &= \frac{2}{r^3}v - \frac{2}{r^2}v_{r} + \frac{1}{r} v_{rr}
.\end{align}
Pluggin this into \eref{sphere-wave-eq} and simplifying we find
\begin{align}
    \label{eq:wave-eq-transformed}
    \left( \frac{v}{r} \right)_{t t} = \frac{1}{r} v_{t t} &= c^2\Biggl[\frac{2}{r^3}v - \frac{2}{r^2}v_{r} + \frac{1}{r}v_{rr} - \frac{2}{r^3}v + \frac{2}{r^2}v_{r} \Biggr] \notag \\
    v_{t t} &= c^2 v_{rr}
,\end{align}
which is the linear 1D wave equation for $v$.  

(b) Solve for $v$ using $v(x,t) = f(x + ct) + g(x-ct)$ and thereby solve the spherical wave equation.

The solution to \eref{wave-eq-transformed} is just
\begin{eqnarray}
    \label{eq:wave-soln-v}
    \eqbox{
    \begin{aligned}
        v &= f(x + ct) + g(x-ct) \\
        u &= \frac{1}{r}[f(x + ct) + g(x-ct)]
    \end{aligned}
    }
.\end{eqnarray}


(c) Use 
\begin{eqnarray}
    \label{eq:gen-sol}
    v(x,t) = \frac{1}{2}\left[ \phi(x + ct) + \phi(x-ct) \right] + \frac{1}{2c}\int_{x-ct}^{x+ct} \psi(s) \dd{s}
,\end{eqnarray}
to solve it with initial conditions $u(r,0) = \phi(r)$, $u_{t}(r,0) = \psi(r)$, taking both $\phi(r)$ and $\psi(r)$ to be even functions of $r$.

The initial conditions on $u$ give the initial conditions $v(r,0) = r\phi(r)$ and $v_{t}(r,0) = r\psi(r)$ so
\begin{eqnarray}
    \label{eq:sphere-soln}
    \eqbox{
    \begin{aligned}
    v(r,t) &= \frac{1}{2}\left[ (r+ct)\phi(r+ct) + (r-ct)\phi(r-ct) \right] + \frac{1}{2c} \int_{r-ct}^{r+ct} s\psi(s) \dd{s} \\
    u(r,t) &= \frac{1}{2r}\left[ (r+ct)\phi(r+ct) + (r-ct)\phi(r-ct) \right] + \frac{1}{2cr} \int_{r-ct}^{r+ct} s\psi(s) \dd{s}
    \end{aligned}
}
.\end{eqnarray}




\end{document}
