\def\duedate{09/29/22}
\def\HWnum{4}
% Document setup
\documentclass[12pt]{article}
\usepackage[margin=1in]{geometry}
\usepackage{fancyhdr}
\usepackage{lastpage}

\pagestyle{fancy}
\lhead{Richard Whitehill}
\chead{MATH 757 -- HW \HWnum}
\rhead{\duedate}
\cfoot{\thepage \hspace{1pt} of \pageref{LastPage}}

% Encoding
\usepackage[utf8]{inputenc}
\usepackage[T1]{fontenc}

% Math/Physics Packages
\usepackage{amsmath}
\usepackage{amssymb}
\usepackage{mathtools}
\usepackage[arrowdel]{physics}
\usepackage{siunitx}

\AtBeginDocument{\RenewCommandCopy\qty\SI}

% Reference Style
\usepackage{hyperref}
\hypersetup{
    colorlinks=true,
    linkcolor=blue,
    filecolor=magenta,
    urlcolor=cyan,
    citecolor=green
}

\newcommand{\eref}[1]{Eq.~(\ref{eq:#1})}
\newcommand{\erefs}[2]{Eqs.~(\ref{eq:#1})--(\ref{eq:#2})}

\newcommand{\fref}[1]{Fig.~\ref{fig:#1}}
\newcommand{\frefs}[2]{Figs.~\ref{fig:#1}--\ref{fig:#2}}

\newcommand{\tref}[1]{Table~\ref{tab:#1}}
\newcommand{\trefs}[2]{Tables~\ref{tab:#1}-\ref{tab:#2}}

% Figures and Tables 
\usepackage{graphicx}
\usepackage{float}

\newcommand{\bef}{\begin{figure}[h!]\begin{center}}
\newcommand{\eef}{\end{center}\end{figure}}

\newcommand{\bet}{\begin{table}[h!]\begin{center}}
\newcommand{\eet}{\end{center}\end{table}}

% tikz
\usepackage{tikz}
\usetikzlibrary{calc}
\usetikzlibrary{decorations.pathmorphing}
\usetikzlibrary{decorations.markings}
\usetikzlibrary{arrows.meta}
\usetikzlibrary{positioning}

% tcolorbox
\usepackage[most]{tcolorbox}
\usepackage{xcolor}
\usepackage{xifthen}
\usepackage{parskip}

\newcommand*{\eqbox}{\tcboxmath[
    enhanced,
    colback=black!10!white,
    colframe=black,
    sharp corners,
    size=fbox,
    boxsep=8pt,
    boxrule=1pt
]}

% Miscellaneous Definitions/Settings
\newcommand{\prob}[2]{\textbf{#1)} #2}
\newcommand{\reals}{\mathbb{R}}
\newcommand{\integers}{\mathbb{Z}}
\newcommand{\naturals}{\mathbb{N}}
\newcommand{\rationals}{\mathbb{Q}}
\newcommand{\complexs}{\mathbb{C}}

\setlength{\parskip}{\baselineskip}
\setlength{\parindent}{0pt}
\setlength{\headheight}{14.49998pt}
\addtolength{\topmargin}{-2.49998pt}







\begin{document}
    
\prob{2.4.16}{
Solve the diffusion equation with constant dissipation: $u_{t} - ku_{xx} + bu = 0$ for $- \infty < x < \infty$ with $u(x,0) = \phi(x)$.
}

Consider the change of variables given by $u = e^{-bt}v$.
Then $u_{t} = -bu + e^{-bt}v_{t}$ and $u_{xx} = e^{-bt}v_{xx}$, which makes the diffusion equation with constant dissipation
\begin{eqnarray}
    \label{eq:heat-with-diss}
    -bu + e^{-bt}v_{t} - ke^{-bt}v_{xx} + bu = e^{-bt}v_{t} - ke^{-bt}v_{xx} = 0 \Rightarrow v_{t} - kv_{xx} = 0
.\end{eqnarray}
That is, $v = e^{bt}u$ satisfies the diffusion equation with no dissipation, where the initial condition $v(x,0) = \phi(x)$
\begin{eqnarray}
    \label{eq:v-constant-diss}
    v = \frac{1}{\sqrt{4 \pi k t}}\int_{-\infty}^{\infty} e^{-(x-y)^2/4kt}\phi(y) \dd{y}
,\end{eqnarray}
which gives $u$ as
\begin{eqnarray}
    \label{eq:u-from-v}
    \eqbox{
    u = \frac{e^{-bt}}{\sqrt{4 \pi k t}}\int_{-\infty}^{\infty} e^{-(x-y)^2/4kt} \phi(y) \dd{y}
}
.\end{eqnarray}
This form of $u$ makes sense.
It is similar to a harmonic oscillator with linear damping.
That is, the solution is a product of the decaying exponential $e^{-bt}$, which depends on the damping parameter $b$, and the solution of the diffusion equation (with no damping). 

\prob{2.4.17}{
Solve the diffusion equation with variable dissipation: $u_{t} - ku_{x} + bt^2u = 0$ for $-\infty < x < \infty$ with $u(x,0) = \phi(x)$.
}

Consider the change of variables given by $u = e^{-bt^3/3}v$.
Thus, $u_{t} = e^{-bt^3/3}v_{t} - bt^2u$ and $u_{xx} = e^{-bt^3/3}v_{xx}$, making the diffusion equation with variable dissipation
\begin{eqnarray}
    \label{eq:heat-with-var-diss}
    e^{-bt^3/3}v_{t} - bt^2u - ke^{-bt^3/3}v_{xx} + bt^2u = e^{-bt^3/3}[v_{t} - kv_{xx}] = 0 \Rightarrow v_{t} - kv_{xx} = 0
.\end{eqnarray}
Hence, observing that $v(x,0) = \phi(x)$
\begin{eqnarray}
    \label{eq:v-var-diss}
    v = \frac{1}{\sqrt{4 \pi k t}} \int_{-\infty}^{\infty} e^{-(x-y)^2/4kt}\phi(y) \dd{y}
.\end{eqnarray}
\begin{eqnarray}
    \label{eq:u-var-diss}
    \eqbox{
    u = \frac{e^{-bt^3/3}}{\sqrt{4 \pi k t}} \int_{-\infty}^{\infty} e^{-(x-y)^2/4kt} \phi(y) \dd{y}
}
,\end{eqnarray}
which is similar to the form of \eref{u-from-v}, except that our solution decays faster since the dissipation becomes quadratically stronger over time.


\prob{2.4.18}{
Solve the heat equation with convection: $u_{t} - ku_{xx} + Vu_{x} = 0$ for $-\infty < x < \infty$ with $u(x,0) = \phi(x)$.
}

We have the equation $u_{t} - Au = 0$ with $A = k\partial_{x}^2 - V\partial_{x}$, which has solution
\begin{eqnarray}
    \label{eq:u-operator}
    u = e^{kt\partial_{x}^2}e^{-Vt\partial_{x}}\phi(x) = e^{-kt\partial_{x}^2}\phi(x-Vt)
,\end{eqnarray}
which means that $\phi(x-Vt)$ satisfies the diffusion equation (without convection).
This makes our solution (letting $y = x - Vt$)
\begin{eqnarray}
    \label{eq:u-with-change}
    \eqbox{
    u = \frac{1}{\sqrt{4 \pi k t}} \int_{-\infty}^{\infty} e^{-(y-z)^2/4kt} \phi(z) \dd{z} = \frac{1}{\sqrt{4 \pi k t}} \int_{-\infty}^{\infty} e^{-(x - Vt - z)^2/4kt}\phi(z) \dd{z}
}
.\end{eqnarray}


\prob{2.5.4}{
Here is a direct relationship between the wave and diffusion equations.
Let $u(x,t)$ solve the wave equation on the whole line with bounded second derivatives.
Let
\begin{eqnarray}
    \label{eq:v-from-u}
    v(x,t) = \frac{c}{\sqrt{4 \pi k t}}\int_{-\infty}^{\infty} e^{-s^2c^2/4kt}u(x,s) \dd{s}
.\end{eqnarray}
}

(a) Show that $v(x,t)$ solves the diffusion equation!

We will show that $v$ satisfies the equation $v_{t} - kv_{xx} = 0$, assuming that $u$ satisfies the equation $u_{tt} = c^2u_{xx}$.
Observe that
\begin{align}
\label{eq:time-deriv-v} 
    \pdv{v}{t} &= -\frac{1}{2t}\frac{c}{\sqrt{4\pi k t}}\int_{-\infty}^{\infty} e^{-s^2c^2/4kt} u(x,s) \dd{s} + \frac{c}{\sqrt{4 \pi k t}} \int_{-\infty}^{\infty} \frac{s^2c^2}{4kt^2}e^{-s^2c^2/4kt}u(x,s) \dd{s} \notag \\
               &= \frac{c}{\sqrt{4 \pi k t}} \int_{-\infty}^{\infty} \Big[ \frac{s^2c^2}{4kt^2} - \frac{1}{2t} \Big]e^{-s^2c^2/4kt} u(x,s) \dd{s}
\end{align}
Observe that 
\begin{eqnarray}
    \label{eq:t-deriv-exp}
    \pdv[2]{s}e^{-s^2c^2/4kt} = \frac{c^2}{4k^2t^2}\Big[ s^2c^2 - 2kt \Big] e^{-s^2c^2/4kt} = \frac{c^2}{k}\Big[ \frac{s^2c^2}{4kt^2} - \frac{1}{2t} \Big]e^{-s^2c^2/4kt}
.\end{eqnarray}
Thus, \eref{time-deriv-v} becomes
\begin{eqnarray}
    \label{eq:time-deriv-v-2}
    \pdv{v}{t} = \frac{c}{\sqrt{4 \pi k t}} \int_{-\infty}^{\infty} \frac{k}{c^2} \pdv[2]{s}\Big(e^{-s^2c^2/4kt} \Big)u(x,s) \dd{s} = \frac{k}{c^2} \frac{c}{\sqrt{4 \pi k t}} \int_{-\infty}^{\infty} e^{-s^2c^2/4kt} u_{ss}(x,s) \dd{s}
.\end{eqnarray}

Next, we have 
\begin{eqnarray}
    \label{eq:space-deriv-v}
    \pdv[2]{x} \frac{c}{\sqrt{4 \pi k t}}\int_{-\infty}^{\infty} e^{-s^2c^2/4kt}u(x,s) \dd{s} = \frac{c}{\sqrt{4 \pi k t}}\int_{-\infty}^{\infty} e^{-s^2c^2/4kt}u_{xx}(x,s) \dd{s}
.\end{eqnarray}

Hence,
\begin{align}
    \label{eq:piecing-puzzle}
    \eqbox{
    v_{t} - k v_{xx} = \frac{ck}{\sqrt{4 \pi k t}} \int_{-\infty}^{\infty} \Big[ \frac{1}{c^2}u_{ss}(x,s) - u_{xx}(x,s) \Big] e^{-s^2c^2/4kt} \dd{s} = 0
}
\end{align}



(b) Show that $\displaystyle \lim_{t \rightarrow 0} v(x,t) = u(x,0)$.

We have already proven that for $\delta > 0$
\begin{eqnarray}
    \label{eq:gaussian-zero-unless-zero}
    \lim_{t \rightarrow 0} \max_{\delta \leq |s| < \infty} \frac{1}{\sqrt{4 \pi k t}}e^{-s^2/4kt} = 0
.\end{eqnarray}
In this case, we let $1/4kt \rightarrow c^2/4kt$ for this problem.
Hence, this tells us that 
\begin{eqnarray}
    \label{eq:gaussian-limit}
    \lim_{t \rightarrow 0} \frac{c}{\sqrt{4 \pi k t}} e^{-s^2c^2/4kt} = 0 ~\mbox{if}~s>0
.\end{eqnarray}
Furthermore, it is trivial to see that if $s=0$, then
\begin{eqnarray}
    \label{eq:limit-at-0}
    \lim_{t \rightarrow 0} v(0,t) = \lim_{t \rightarrow 0} \frac{c}{\sqrt{4 \pi k t}} = \infty
.\end{eqnarray}

Thus, we can write
\begin{eqnarray}
    \label{eq:gaussian-times-function}
    \lim_{t \rightarrow 0} \frac{c}{\sqrt{4 \pi k t}} e^{-s^2c^2/4kt}u(x,s) = u(x,0) \lim_{t \rightarrow 0} \frac{c}{\sqrt{4 \pi k t}} e^{-s^2c^2/4kt}
.\end{eqnarray}
We can now prove the claim of this problem:
\begin{eqnarray}
    \label{eq:limit-v}
    \eqbox{
    \begin{aligned}
    \lim_{t \rightarrow 0} v(x,t) &= \lim_{t \rightarrow 0} \int_{-\infty}^{\infty} \frac{c}{\sqrt{4 \pi k t}} e^{-s^2 c^2/4kt}u(x,s) \dd{s} \notag \\
    &= u(x,0) \lim_{t \rightarrow 0} \int_{-\infty}^{\infty} \frac{c}{\sqrt{4 \pi k t}} e^{-s^2c^2/4kt} \dd{s} = u(x,0)
    \end{aligned}
    }
,\end{eqnarray}
since $c\exp(-s^2c^2/4kt)/\sqrt{4 \pi k t}$ is a normalized gaussian for all $t>0$.
That is,
\begin{eqnarray}
    \label{eq:normalized-gaussian}
    \int_{-\infty}^{\infty} \frac{c}{\sqrt{4 \pi k t}} e^{-s^2c^2/4kt} \dd{s} = 1
.\end{eqnarray}



\prob{12.3}{
Check formulas (6) and (7) from page 345:
\begin{align}
    \label{eq:formula-6} 
    \mathcal{F}\{H(a - |x|)\} &= \frac{2}{k}\sin{ak} \\
    \label{eq:formula-7}
    \mathcal{F}\{ e^{-a|x|} \} &= \frac{2a}{a^2 + k^2} \quad \mbox{for}~ a > 0
,\end{align}
where $\displaystyle \mathcal{F}\{f\} = \int_{-\infty}^{\infty} f(x) e^{-ikx} \dd{x}$ and $H(x)$ is the Heaviside step function.
}

For \eref{formula-6}, we have 
\begin{eqnarray}
    \label{eq:fourier-heaviside}
    \eqbox{
    \mathcal{F}\{ H(a-|x|) \} = \int_{-\infty}^{\infty} H(a-|x|)e^{-ikx} \dd{x} = \int_{-a}^{a} e^{-ikx} \dd{x} = \frac{e^{-ika} - e^{ika}}{-ik} = \frac{2}{k}\sin{ka}
    }
,\end{eqnarray}
where we have used the identity
\begin{eqnarray}
    \label{eq:sine-exp-id}
    \sin{x} = \frac{e^{ikx} - e^{-ikx}}{2i}
.\end{eqnarray}

Now, for \eref{formula-7}, we have
\begin{align}
    \label{eq:fourier-exp-abs-x}
    \mathcal{F}\{e^{-a|x|}\} &= \int_{-\infty}^{\infty} e^{-a|x|}e^{-ikx} \dd{x} = \int_{-\infty}^{0} e^{(a-ik)x} \dd{x} + \int_{0}^{\infty} e^{-(a+ik)x} \dd{x} \\
    &= \frac{1}{a-ik} + \frac{1}{a+ik}
,\end{align}
where we have assumed $a > 0$ such that the evaluation of the integral vanishes at $\pm \infty$.
Collecting terms and simplifying we find
\begin{eqnarray}
    \label{eq:rewrite-formula-7}
    \eqbox{
    \mathcal{F}\{e^{-a|x|}\} = \frac{(a+ik) + (a-ik)}{(a-ik)(a+ik)} = \frac{2a}{a^2 + k^2}
}
.\end{eqnarray}


\prob{6}{
Check formulas (i), (ii), and (iii) on page 346:
\begin{align}
    \label{eq:formula-i} 
    &\mbox{i}:~\mathcal{F}\Big\{\dv{f}{x}\Big\} = ik\mathcal{F}\{f(x)\} \\
    \label{eq:formula-ii}
    &\mbox{ii}:~ \mathcal{F}\{xf(x)\} = i\dv{\mathcal{F}\{f(x)\}}{k} \\
    \label{eq:formula-iii}
    &\mbox{iii}:~ \mathcal{F}\{f(x-a)\} = e^{-iak}\mathcal{F}\{f(x)\}
\end{align}
}

For \eref{formula-i} we have
\begin{eqnarray}
    \label{eq:formula-i-work}
    \eqbox{
    \mathcal{F}\Big\{\dv{f}{x}\Big\} = \int_{-\infty}^{\infty} \dv{f}{x} e^{-ikx} \dd{x} = -ike^{-ikx}f(x)\Big|_{-\infty}^{\infty} + ik \int_{-\infty}^{\infty} f(x) e^{-ikx} \dd{x} = ik\mathcal{F}\{f(x)\}
}
,\end{eqnarray}
where we have supposed that $f$ goes to zero at $\pm \infty$.

Next, for \eref{formula-ii} we have
\begin{eqnarray}
    \label{eq:formula-ii-work}
    \mathcal{F}\{xf(x)\} = \int_{-\infty}^{\infty} xf(x)e^{-ikx} \dd{x} = \int_{-\infty}^{\infty}f(x)\Big(\frac{1}{-i}\Big)\pdv{k}e^{-ikx} \dd{x} \\
    = \eqbox{i\dv{k}\int_{-\infty}^{\infty} f(x)e^{-ikx} \dd{x} = i\dv{\mathcal{F}\{f(x)\}}{k}
}
.\end{eqnarray}

Finally, for \eref{formula-iii} we can make the substitution $y = x-a$ such that
\begin{eqnarray}
    \label{eq:formula-iii-work}
    \eqbox{
    \mathcal{F}\{f(x-a)\} = \int_{-\infty}^{\infty} f(x-a)e^{-ikx} \dd{x} = \int_{-\infty}^{\infty} f(y)e^{-iky}e^{-ika} \dd{y} = e^{-ika}\mathcal{F}\{f(x)\}
    }
.\end{eqnarray}





\end{document}
