\def\duedate{09/15/2022}
\def\HWnum{3}
% Document setup
\documentclass[12pt]{article}
\usepackage[margin=1in]{geometry}
\usepackage{fancyhdr}
\usepackage{lastpage}

\pagestyle{fancy}
\lhead{Richard Whitehill}
\chead{MATH 757 -- HW \HWnum}
\rhead{\duedate}
\cfoot{\thepage \hspace{1pt} of \pageref{LastPage}}

% Encoding
\usepackage[utf8]{inputenc}
\usepackage[T1]{fontenc}

% Math/Physics Packages
\usepackage{amsmath}
\usepackage{amssymb}
\usepackage{mathtools}
\usepackage[arrowdel]{physics}
\usepackage{siunitx}

\AtBeginDocument{\RenewCommandCopy\qty\SI}

% Reference Style
\usepackage{hyperref}
\hypersetup{
    colorlinks=true,
    linkcolor=blue,
    filecolor=magenta,
    urlcolor=cyan,
    citecolor=green
}

\newcommand{\eref}[1]{Eq.~(\ref{eq:#1})}
\newcommand{\erefs}[2]{Eqs.~(\ref{eq:#1})--(\ref{eq:#2})}

\newcommand{\fref}[1]{Fig.~\ref{fig:#1}}
\newcommand{\frefs}[2]{Figs.~\ref{fig:#1}--\ref{fig:#2}}

\newcommand{\tref}[1]{Table~\ref{tab:#1}}
\newcommand{\trefs}[2]{Tables~\ref{tab:#1}-\ref{tab:#2}}

% Figures and Tables 
\usepackage{graphicx}
\usepackage{float}

\newcommand{\bef}{\begin{figure}[h!]\begin{center}}
\newcommand{\eef}{\end{center}\end{figure}}

\newcommand{\bet}{\begin{table}[h!]\begin{center}}
\newcommand{\eet}{\end{center}\end{table}}

% tikz
\usepackage{tikz}
\usetikzlibrary{calc}
\usetikzlibrary{decorations.pathmorphing}
\usetikzlibrary{decorations.markings}
\usetikzlibrary{arrows.meta}
\usetikzlibrary{positioning}

% tcolorbox
\usepackage[most]{tcolorbox}
\usepackage{xcolor}
\usepackage{xifthen}
\usepackage{parskip}

\newcommand*{\eqbox}{\tcboxmath[
    enhanced,
    colback=black!10!white,
    colframe=black,
    sharp corners,
    size=fbox,
    boxsep=8pt,
    boxrule=1pt
]}

% Miscellaneous Definitions/Settings
\newcommand{\prob}[2]{\textbf{#1)} #2}
\newcommand{\reals}{\mathbb{R}}
\newcommand{\integers}{\mathbb{Z}}
\newcommand{\naturals}{\mathbb{N}}
\newcommand{\rationals}{\mathbb{Q}}
\newcommand{\complexs}{\mathbb{C}}

\setlength{\parskip}{\baselineskip}
\setlength{\parindent}{0pt}
\setlength{\headheight}{14.49998pt}
\addtolength{\topmargin}{-2.49998pt}







\begin{document}
    
\prob{2.4.3}{
Use 
\begin{eqnarray}
    \label{eq:source-func}
    u(x,t) = \frac{1}{\sqrt{4\pi kt}}\int_{-\infty}^{\infty} e^{-(x-y)^2/4kt} \phi(y) \dd{y}
.\end{eqnarray}
to solve the diffusion equation if $\phi(x) = e^{3x}$.
}

Plugging $\phi$ into \eref{source-func}, we have
\begin{align}
    \label{eq:source-with-phi}
    u(x,t) &= \frac{1}{\sqrt{4 \pi k t}} \int_{-\infty}^{\infty} e^{-(x-y)^2/4kt + 3y} \dd{y} \notag \\
           &= \frac{1}{\sqrt{4 \pi k t}} \int_{-\infty}^{\infty} e^{-(x^2-2xy+y^2-12kty)/4kt} \dd{y} \notag \\
           &= \frac{1}{\sqrt{4 \pi k t}} e^{-x^2/4kt} \int_{-\infty}^{\infty} e^{-(y^2-2(6kt+x)y+(6kt-x)^2 - (6kt-x)^2)/4kt} \dd{y} \notag \\
           &= \frac{1}{\sqrt{4 \pi k t}} e^{-x^2/4kt}e^{(x-6kt)^2/4kt} \int_{-\infty}^{\infty} e^{-[y-(6kt+x)]^2/4kt} \dd{y} \notag \\
           &= \eqbox{e^{3x + 9kt}}
.\end{align}



\prob{2.4.6}{
    Compute $\int_{0}^{\infty} e^{-x^2} \dd{x}$.
}

Let
\begin{eqnarray}
    \label{eq:I-def}
    I = \int_{0}^{\infty} e^{-x^2} \dd{x}
,\end{eqnarray}
then
\begin{eqnarray}
    \label{eq:Isq}
    I^2 = \int_{0}^{\infty}\int_{0}^{\infty} e^{-x^2}e^{-y^2} \dd{x}\dd{y} = \int_{0}^{\infty}\int_{0}^{\infty} e^{-(x^2+y^2)} \dd{x}\dd{y}
.\end{eqnarray}
Changing to polar coordinates, we have $\dd{x}\dd{y} = r\dd{r}\dd{\phi}$, where $r \in [0,\infty]$ and $\phi \in [0,\pi/2]$ and $r = \sqrt{x^2 + y^2}$.
Thus, \eref{Isq} becomes
\begin{eqnarray}
    \label{eq:Isq-polar}
    I^2 = \int_{0}^{\pi/2} \int_{0}^{\infty} e^{-r^2} r\dd{r}\dd{\phi}
.\end{eqnarray}
If we make the substitution $u = r^2$, then
\begin{eqnarray}
    \label{eq:Isq-sub}
    I^2 = \frac{\pi}{4} \int_{0}^{\infty} e^{-u} \dd{u} = \frac{\pi}{4}
,\end{eqnarray}
leaving us with
\begin{eqnarray}
    \label{eq:I-result}
    \eqbox{
    I = \sqrt{\frac{\pi}{4}} = \frac{\sqrt{\pi}}{2}
}
.\end{eqnarray}

\prob{2.4.7}{
    Use Exercise 6 to show that $\int_{-\infty}^{\infty} e^{-p^2} \dd{p} = \sqrt{\pi}$.
    Then substitute $p = x/\sqrt{4kt}$ to show that 
    \begin{eqnarray}
        \label{eq:result-7}
        \int_{-\infty}^{\infty} S(x,t) \dd{x} = 1
    .\end{eqnarray}
}

Notice that $e^{-p^2}$ is an even function in $p$, so
\begin{eqnarray}
    \label{eq:gauss-int}
    \int_{-\infty}^{\infty} e^{-p^2} \dd{p} = 2\int_{0}^{\infty} e^{-p^2} \dd{p} = 2\frac{\sqrt{\pi}}{2} = \sqrt{\pi}
.\end{eqnarray}
It is given in the text that $S(x,t) = \exp(-x^2/4kt)/\sqrt{4 \pi k t}$, so
\begin{eqnarray}
    \label{eq:int-source}
    \int_{-\infty}^{\infty} S(x,t) \dd{x} = \frac{1}{\sqrt{4 \pi k t}}\int_{-\infty}^{\infty} e^{-\frac{x^2}{4kt}} \dd{x}
.\end{eqnarray}
If we make the change of variables $p = x/\sqrt{4kt}$, then $p \in (-\infty,\infty)$ and $\dd{x} = \sqrt{4 k t}\dd{p}$, meaning
\begin{eqnarray}
    \label{eq:int-source-sub}
    \eqbox{
    \int_{-\infty}^{\infty} S(x,t) \dd{x} = \frac{1}{\sqrt{\pi}} \int_{-\infty}^{\infty} e^{-p^2} = 1
}
.\end{eqnarray}


\prob{2.4.8}{
Show that for any fixed $\delta > 0$ (no matter how small),
\begin{eqnarray}
    \label{eq:result-8}
    \max_{\delta \leq |x| < \infty} S(x,t) \rightarrow 0 \quad {\rm as}~ t \rightarrow 0
.\end{eqnarray}
}

Observe that $S(x,t)$ is a monotonically decreasing function for $x > 0$ at every $t$.
Thus, the absolute maximum of $S(x,t)$ is taken on whenever $x = \delta$ in the interval $[\delta,\infty)$.
That is,
\begin{eqnarray}
    \label{eq:max-S}
    \lim_{t \rightarrow 0} \max_{\delta \leq |x| < \infty} S(x,t) = \lim_{t \rightarrow 0} S(\delta,t) = \lim_{t \rightarrow 0} \frac{1}{\sqrt{4 \pi k t}}e^{-\delta^2/4 k t}
.\end{eqnarray}
We make the change of variables $s = 1/4kt$, which makes \eref{max-S}
\begin{eqnarray}
    \label{eq:limit}
    \eqbox{
    \lim_{t \rightarrow 0} \frac{e^{-\delta^2/4kt}}{\sqrt{4 \pi k t}} = \frac{1}{\sqrt{\pi}}\lim_{s \rightarrow \infty} \frac{\sqrt{s}}{e^{\delta^2 s}} = \frac{1}{2\delta^2\sqrt{\pi}} \lim_{s \rightarrow \infty} \frac{1}{\sqrt{s}e^{\delta^2 s}} = 0
}
,\end{eqnarray}
where L'Hopital's rule was used.


\prob{2.4.9}{
Solve the diffusion equation $u_{t} = ku_{xx}$ with the initial condition $u(x,0) = x^2$ by the following special method.
First show that $u_{xxx}$ satisfies the diffusion equation with \textit{zero} initial condition.
Therefore, by uniqueness $u_{xxx} \equiv 0$.
Integrating this result thrice, obtain $u(x,t) = A(t)x^2 + B(t)x + C(t)$.
Finally, it's easy to solve for $A$, $B$, and $C$ by plugging into the original problem.
}

Notice 
\begin{eqnarray}
    \label{eq:derivative-heat-eq}
    (u_{xxx})_{t} = (u_{t})_{xxx} = (ku_{xx})_{xxx} = k(u_{xxx})_{xx}
.\end{eqnarray}
Hence, $u_{xxx}$ is a solution to the diffusion equation and satisfies $u_{xxx}(x,0) = \partial_{x}^{3}(x^2) = 0$.
By uniqueness, it immediately follows that $u_{xxx} \equiv 0$, and upon integration, we obtain
\begin{eqnarray}
    \label{eq:uxxx-int}
    u = A(t)x^2 + B(t)x + C
.\end{eqnarray}
We can then solve for $A$, $B$, and $C$ by plugging our expression for $u$ into the diffusion equation.
\begin{eqnarray}
    \label{eq:solve-ABC}
    A'(t)x^2 + B'(t)x + C'(t) = 2kA(t)
.\end{eqnarray}
Matching the coefficients in $x$, we have
\begin{eqnarray}
    \label{eq:eq-ABC}
    \begin{cases}
    A' = 0 \\
    B' = 0 \\
    C' = 2kA
    \end{cases} 
.\end{eqnarray}
Thus, $A = c_{1}$, $B = c_{2}$, and $C = 2kc_{1}t + c_{3}$, giving
\begin{eqnarray}
    \label{eq:u_xt}
    u(x,t) = c_{1}x^2 + c_{2}x + 2kc_{1}t + c_{3}
.\end{eqnarray}
Finally, we can solve for our coefficients by plugging in our initial condition 
\begin{eqnarray}
    \label{eq:solve-coeff}
    u(x,0) = x^2 = c_{1}x^2 + c_{2}x + c_{3} 
,\end{eqnarray}
which makes our solution
\begin{eqnarray}
    \label{eq:u_xt-sol}
    \eqbox{
    u(x,t) = x^2 + 2kt
}
.\end{eqnarray}


\prob{2.4.15}{
Prove the uniqueness of the diffusion problem with Neumann boundary conditions:
\begin{align}
    \label{eq:Neumann-BC}
    u_{t} - ku_{xx} = f(x,t) \quad {\rm for}~ 0 < x < l,t > 0 \\
    u(x,0) = \phi(x) ~~ u_{x}(0,t) = g(t) ~~ u_{x}(l,t) = h(t)
.\end{align}
by the energy method.
}

Suppose that there are two distinct solutions to the diffusion problem satisfying Neumann boundary conditions: $u$ and $v$.
We may define $w = u - v$, which is a solution to the diffusion problem with the following boundary conditions: $w(x,0) = 0$, $w_{x}(0,t) = 0$, and $w_{x}(l,t)0$.

Now, consider the following quantity
\begin{eqnarray}
    \label{eq:energy}
    E(t) = \int_{0}^{l} w^2(x,t) \dd{x}
.\end{eqnarray}
We may differentiate this expression, yielding
\begin{eqnarray}
    \label{eq:diff-energy}
    \dv{E}{t} = \int_{0}^{l} 2ww_{t} \dd{x} = \int_{0}^{l} 2w(kw_{xx}) \dd{x} 
.\end{eqnarray}
We may integrate by parts, noticing that the boundary term vanishes since $w_{x}$ vanishes at the boundaries, giving
\begin{eqnarray}
    \label{eq:diff-energy-int-by-parts}
    \dv{E}{t} = -2k\int_{0}^{l} w_{x}^2 \dd{x} \leq 0
,\end{eqnarray}
since the integrand is positive-definite.
This, equation says that $E(t)$ is a strictly decreasing function of $t$, and since $E(0) = 0$, it follows that $E(t) \leq 0$.
We also know by inspection of \eref{energy}, that $E(t) \geq 0$ for all $t$.
Hence, both inequalities may only be satisfied simultaneously if $E(t) = 0$, which implies that the integrand must be identically zero and 
\begin{eqnarray}
    \label{eq:uniqueness}
    \eqbox{u \equiv v}
.\end{eqnarray}







\end{document}
