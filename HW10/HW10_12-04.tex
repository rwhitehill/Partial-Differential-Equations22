\def\duedate{12/04/22}
\def\HWnum{10}
\input{../preamble.tex}

\begin{document}
    
\prob{12.1.5}{
Verify, directly from the definition of a distribution, that the discontinuous function $u(x,t) = H(x - ct)$ is a weak solution of the wave equation.
}

We must check that $u$ satisfies the wave equation in the sense that $\langle u_{tt} - c^2u_{xx},\phi \rangle = 0$, where $\phi$ is any test function.
We have
\begin{eqnarray}
    \langle u,\phi \rangle = \int_{-\infty}^{\infty} \int_{-\infty}^{\infty} u(x,t) \phi(x,t) \dd{x} \dd{t} = \int_{-\infty}^{\infty} \int_{ct}^{\infty} \phi(x,t) \dd{x} \dd{t} = \int_{-\infty}^{\infty} \int_{-\infty}^{x/c} \phi(x,t) \dd{t} \dd{x}
.\end{eqnarray}
Note that the last two integrations are equivalent with only the order of integration swapped.
Thus,
\begin{eqnarray}
    \langle u_{tt},\phi \rangle = \langle u,\phi_{tt} \rangle = \int_{-\infty}^{\infty} \int_{-\infty}^{x/c} \phi_{tt}(x,t) \dd{x} \dd{t} = \int_{-\infty}^{\infty} \phi_{t}(x,x/c) \dd{x}
,\end{eqnarray}
where we have used the fact that the test functions we satisfy $\lim_{x \rightarrow \pm \infty} \phi(x) = 0$ (for the one-dimensional case).
Now we will introduce the substitution $y = x/c$ to rewrite the integral above
\begin{eqnarray}
    \langle u_{t t},\phi \rangle = \int_{-\infty}^{\infty} c \phi_{t}(cy,y) \dd{y}
.\end{eqnarray}

Similarly,
\begin{eqnarray}
    \langle u_{xx},\phi \rangle = \langle u,\phi_{xx} \rangle = \int_{-\infty}^{\infty} \int_{ct}^{\infty} \phi_{xx}(x,t) \dd{x} \dd{t} = -\int_{-\infty}^{\infty} \phi_{x}(ct,t) \dd{t}
.\end{eqnarray}
Hence,
\begin{eqnarray}
    \eqbox{
    \langle u_{tt} - c^2 u_{xx},\phi \rangle = \int_{-\infty}^{\infty} [c \phi_{t}(cy,y) + c^2 \phi_{x}(cy,y)] \dd{y} = \int_{-\infty}^{\infty} c \dv{\phi(cy,y)}{y} \dd{y} = 0
}
,\end{eqnarray}
where $y$ depends implicitly on $t$.

\prob{12.3.4}{
Prove the following properties of the convolution
\begin{eqnarray}
    (f * g)(x) = \int_{-\infty}^{\infty} f(x - y) g(y) \dd{y}
.\end{eqnarray}
}

a) $f * g = g * f$

We make a change of variables in the integration $z = x - y$, which gives
\begin{eqnarray}
    \eqbox{
    f * g = -\int_{\infty}^{-\infty} f(z) g(x - z) \dd{z} = g * f
}
.\end{eqnarray}


b) $(f * g)' = f' * g = f * g'$, where $'$ denotes the derivative in one variable.

The derivative of the convolution
\begin{eqnarray}
\eqbox{
\begin{aligned}    
    (f * g)' &= \dv{x} \int_{-\infty}^{\infty} f(x - y) g(y) \dd{y} = \int_{-\infty}^{\infty} \Big[ \dv{f}{(x-y)} \dv{(x-y)}{x} \Big] g(y) \dd{y} \\
             &= \int_{-\infty}^{\infty} f'(x-y)g(y) \dd{y} = f' * g \\
             &= (g * f)' = g' * f = f * g'
\end{aligned}
}
.\end{eqnarray}


\prob{12.3.5}{}

a) Show that $\delta * f = f$ for any distribution $f$, where $\delta$ is the delta function.

Since $f$ is a distribution, we show this in the sense that $\langle \delta * f, \phi \rangle = \langle f,\phi \rangle$, where $\phi$ is a test function which goes to zero as $|x| \rightarrow \infty$.
The work is as follows:
\begin{eqnarray}
    \eqbox{
\begin{aligned} 
    \langle \delta * f,\phi \rangle &= \int_{-\infty}^{\infty} (\delta * f)(x) \phi(x) \dd{x} = \int_{-\infty}^{\infty} \Bigg[ \int_{-\infty}^{\infty} \delta(x - y) f(y) \dd{y} \Bigg] \phi(x) \dd{x} \\
                                    &= \int_{-\infty}^{\infty} f(x) \phi(x) \dd{x} = \langle f,\phi \rangle
\end{aligned}
}
.\end{eqnarray}

b) Show that $\delta' * f = f'$ for any distribution $f$, where $'$ is the derivative.

Observe that
\begin{eqnarray}
    \eqbox{
    \delta' * f = (\delta * f)' = f'
}
.\end{eqnarray}


\prob{12.3.6}{
Let $f(x)$ be a continuous function defined for $-\infty < x < \infty$ such that its Fourier transform $F(k)$ satisfies $F(k) = 0$ for $|k| > \pi$.
Such a function is said to be \textit{band-limited}
}

a) Show that 
\begin{eqnarray}
    f(x) = \sum_{n=-\infty}^{\infty} f(n) \frac{\sin{[\pi (x-n)]}}{\pi(x-n)}
.\end{eqnarray}
Thus $f(x)$ is completely determined by its values at the integers!
We say that $f(x)$ is \textit{sampled} at the integers.

We can write
\begin{eqnarray}
    f(x) = \int_{-\infty}^{\infty} F(k) e^{ikx} \frac{\dd{k}}{2\pi} = \int_{-\pi}^{\pi} F(k) e^{ikx} \frac{\dd{k}}{2\pi}
.\end{eqnarray}
Utilizing the Fourier series, we can rewrite $F(k)$ as
\begin{eqnarray}
    F(k) = \sum_{n=-\infty}^{\infty} \hat{F}_n e^{-ink}
.\end{eqnarray}
Note that the coefficients
\begin{eqnarray}
    \hat{F}_{n} = \frac{1}{2\pi} \int_{-\pi}^{\pi} F(k) e^{-ink} \dd{k} = f(n)
,\end{eqnarray}
which is the inverse fourier transform of $f$ evaluated at $x = n$.
Hence,
\begin{eqnarray}
    f(x) = \sum_{n=-\infty}^{\infty} \frac{f(n)}{2\pi} \int_{-\pi}^{\pi} e^{-ink} e^{ikx} \dd{k}
.\end{eqnarray}
Performing the integral we find
\begin{eqnarray}
    \int_{-\pi}^{\pi} e^{i(x-n)k} \dd{k} = \frac{2 \sin{[\pi(x-n)]}}{x-n}
,\end{eqnarray}
and inserting this back into our expression above for $f(x)$ we obtain
\begin{eqnarray}
    \eqbox{
    f(x) = \sum_{n=-\infty}^{\infty} f(n) \frac{\sin{[\pi(x-n)]}}{\pi(x-n)}
}
.\end{eqnarray}


b) Let $F(k) = 1$ in the interval $(-\pi,\pi)$ and $F(k) = 0$ outside this interval.
Calculate both sides of $(a)$ directly to verify that they are equal.

We can calculate $f$ from the inverse fourier transform
\begin{eqnarray}
    f(x) = \frac{1}{2\pi}\int_{-\pi}^{\pi} e^{ikx} \dd{k} = \frac{\sin{(\pi x)}}{\pi x} = 
,\end{eqnarray}
and
\begin{eqnarray}
    \sum_{n=-\infty}^{\infty} \frac{\sin{(\pi n)}}{\pi n} \frac{\sin{[\pi(x - n)]}}{\pi(x-n)} = \frac{\sin{(\pi x)}}{\pi x}
,\end{eqnarray}
observing that $\sin{(\pi n)} = 0$ for $n \in \integers \backslash \{ 0 \} $.


\end{document}
